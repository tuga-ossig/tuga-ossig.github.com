\documentclass[a4paper]{article}

\usepackage[utf8]{inputenc}
\usepackage[T1]{fontenc}
\usepackage[ngerman]{babel}

\usepackage{amsmath}
\usepackage{amssymb}

\usepackage{hyperref}

\author{Anon Ymous}
\date{\today}

\title{fourth example}


\begin{document}

\maketitle

\tableofcontents

\section{Math modes}

When you want to add some mathematical notations containing mathematical symbols
you'll need to write it in a special environment because LaTeX typesets maths
notation differently than normal text. There are two main environments you need
to know about.


\subsection{Inline math mode}

The math mode where formulas are displayed inline, in the middle of running
text. Your mathematical formulas need to be enclosed in single dollar
signs, e.g. $x^4 + x^2 = 0$

This inline formula $x = 4$ appears in the flow with the text.


\subsection{Display math mode}

The \textbf{Display} math mode where formulas are displayed centered on a new
line. Your mathematical formulas need to be enclosed in double dollar
signs, e.g. $$x^4 + x^2 = 0$$

This display formula $$x = 4$$ appears out of the text flow.

Of course formulas can be much more complex whether inline in text math mode
$\forall x \in X, \quad \exists y \leq \epsilon$ or in display math mode:
$$\forall x \in X, \quad \exists y \leq \epsilon$$

You can read more about these modes in the LaTeX wikibook at
\url{http://en.wikibooks.org/wiki/LaTeX/Mathematics#Mathematics_environments}


\section{Superscript and subscript}

Subscripts in latex are accomplished with the underscore $x_4 = 1$.
Superscripts in latex are accomplished with the circumflex character
(also called the hat character) $x^4 = 1$.

However, that alone only works with one character after the underscore or hat
$x_40 = 1$.
The solution to this problem is to include the subscripted or superscripted
text in curly braces $x_{40} = 1$.


\section{Fractions}

In order to insert fractions in your text, you have the choice between the
commands \verb`\frac` and \verb`\dfrac`.

This is how the \verb`\frac` command renders in math mode: $\frac{x}{2}$

This is how the \verb`\frac` command renders in display mode: $$\frac{x}{2}$$

You'll notice that depending on the environment fractions render differently. In
case you want to have the display mode rendering in inline math mode, you can
use the \verb`\dfrac` to do so: $\dfrac{x}{2}$. Using it in diplay mode however
doesn't change anything and things render like previously: $$\dfrac{x}{2}$$


\end{document}
