\documentclass[a4paper]{article}

\usepackage[utf8]{inputenc}
\usepackage[T1]{fontenc}
\usepackage[ngerman]{babel}

\usepackage{amsmath}
\usepackage{amssymb}

\usepackage{hyperref}

\author{Anon Ymous}
\date{\today}

\title{Basic math notations}

\begin{document}

\maketitle
\tableofcontents

\section{Arrows}

Arrows are used in maths a lot, but not only. We often need to insert arrow
symbols in our documents, fortunately for us latex makes it easy through the
math modes, and the plethora of arrow symbols available for use.

\subsection{Single arrows}

Those are single arrows:
$$x \rightarrow y$$
$$x \leftarrow y$$
$$x \leftrightarrow y$$


The negated equivalents here are (arrows with stroke):

$$x \nrightarrow y$$
$$x \nleftarrow y$$
$$x \nleftrightarrow y$$

\subsection{Double arrows}

Those are single arrows:
$$x \Rightarrow y$$
$$x \Leftarrow y$$
$$x \Leftrightarrow y$$


The negated equivalents here are (arrows with stroke):
$$x \nRightarrow y$$
$$x \nLeftarrow y$$
$$x \nLeftrightarrow y$$

\subsection{Arrows with a semantic name}

\verb`\to` is a shorthand of \verb`\rightarrow` except with a semantic command
name. This is specially useful in math formulas:

$$U:\mathbb R \to \mathbb R$$

\verb`\gets` is a shorthand of \verb`\leftarrow` except with a semantic command
name.

\verb`\mapsto` is a rightwards arrow from bar, with a semantic command name.

$$x \mapsto x^3 + 1$$


\section{Displaystyle and Summing}

In order for certain notations, such as \verb`\lim` or \verb`\sum` to be
displayed correctly inside some math environments (derrived from the inline
math mode), it might be convenient to wrap your formulas with the
\verb`\displaystyle` class. Doing so might cause the line to be taller, but
will cause exponents and indices to be displayed correctly.

For example the sum notation inline looks like this: $\sum_{i = 0}^{k}{x(i)}$

Whereas in display math mode looks like this:
$$\sum_{i = 0}^{k}{x(i)}$$

The displaymath style in normal math mode (inline):
$\displaystyle{\sum_{i = 0}^{k}{x(i)}}$

\section{Vectors}

Adding a vector notation (the arrow over a letter), is simply accomplished
with the \verb`\vec` command:

$$(\vec{i}, \vec{j})$$

However, the proportions on $\vec{AB}$ are lost. To remedy to this problem,
it's possible to use another command \verb`\overrightarrow` at the cost of the
command's semantic name.

$$\overrightarrow{AB} + \overrightarrow{BC} = \overrightarrow{AC}$$


\end{document}
