\documentclass[a4paper]{report}

\usepackage[utf8]{inputenc}
\usepackage[T1]{fontenc}
\usepackage[ngerman]{babel}

\usepackage{hyperref}

\author{Anon Ymous}
\date{}

\title{third example}


\begin{document}

\maketitle

\tableofcontents

\chapter{first section}

\section{Document}

\subsection{The document's class}

The first example we saw was an \textit{article}, the second was a
\textit{book}, and this example is a \textit{report}.
Different document classes get rendered differently.

There is a very good explanation of the major document classes in the LaTeX
wikibook at \url{http://en.wikibooks.org/wiki/LaTeX/Basics#Document_Classes}.

One this to keep in mind though, is that different document classes have
different document structures, i.e. sectioning commands, check
\url{http://en.wikibooks.org/wiki/LaTeX/Document_Structure#Sectioning_Commands}
for more information.

For example, you cannot have chapters in articles, because it doesn't make
sense there semantically speaking.


\subsection{The document's date}

In the first two examples, a date is rendered on the title page. The first
example however doesn't have an explicit date command in the
\textbf{top matter}.
By default the date is set to the \verb`\today` command's value, which is
today's date.

There is however a difference between the first and second example, and that is
the date's format. The culprit here is the \textit{babel} package which also
changes the date's format so that it conforms with the document's language.

Nevertheless, there comes a time where you don't want a date on your document
and that can be accomplish by defining a date without any parameters, like in
this example.


\section{Internet links}

Looking at the preamble of this document, you'll see we use a new package
\textit{hyperref}. This allows us to insert all sorts of links into our
document. For now we'll only use the \verb`\url` command. It allows you to
insert the link to a web resource in text. Clicking it will launch it in your
default browser.


\chapter{Give me some lists!}

\section{Different kinds of lists}

In latex there are three kinds of lists, we'll have a look at all three types.

\subsection{Itemize}

An unordered list in latex is created with the \textbf{itemize} environment:

\begin{itemize}
    \item First item
    \item Second item
    \item Third item
\end{itemize}

An unordered list in latex is created with the \textbf{itemize} environment.
The \textbf{itemize} environment is semantically suitable for lists of items
for which the order is not important.


\subsection{Enumerate}

An ordered list in latex is created with the \textbf{enumerate} environment:

\begin{enumerate}
    \item First item
    \item Second item
    \item Third item
\end{enumerate}

An ordered list in latex is created with the \textbf{enumerate} environment.
The \textbf{enumerate} environment is semantically suitable for lists of items
for which the order is important.


\subsection{Descriptions}

A definition list in latex is created with the \textbf{description} environment:

\begin{description}
    \item[First] First item
    \item[Second] Second item
    \item[Third] Third item
\end{description}

A definition list in latex is created with the \textbf{description} environment.
The \textbf{description} environment is semantically suitable for lists of
pairs made of terms to be described or defined, and their description or
definition.


\section{Nested lists}

In LaTeX you can insert a list environment into an existing one (up to a depth
of four). Different types of lists can be nested at different levels.

\begin{enumerate}
    \item First item
    \item First level nested list
    \begin{itemize}
        \item First item of second level list
        \item Second level nested list
        \begin{description}
            \item[First] First item
            \item[second] Second item
        \end{description}
        \item Third item
    \end{itemize}
    \item last item in list
\end{enumerate}


\end{document}
