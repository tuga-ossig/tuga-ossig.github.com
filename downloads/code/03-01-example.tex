\documentclass[a4paper]{article}

\usepackage[utf8]{inputenc}
\usepackage[T1]{fontenc}
\usepackage[ngerman]{babel}

\usepackage{amsmath}
\usepackage{amssymb}

\usepackage{hyperref}

\author{Anon Ymous}
\date{\today}

\title{Hyperlinks}

\begin{document}

\maketitle
\tableofcontents

\section{Basic}
\LaTeX enables typesetting of hyperlinks through a package called
\textbf{hyperref}.
This works with pdflatex which is what we've been using so far.

If you use it, you will be able to include interactive links and all your
intra-document references (like the table of contents) will be turned into
hyperlinks.

All you need to do in order to start playing with hyperlinks in latex, is to
include \verb`\usepackage{hyperref}` in the preamble.

\subsection{URL}
\url{http://en.wikibooks.org/wiki/LaTeX/Hyperlinks}

It's often useful in documents to include a reference to other documents online.
You can pass your link as an argument of the \verb`\url` command, i.e.
\verb`\url{http://en.wikibooks.org/wiki/LaTeX/Hyperlinks}` and it does
exactly that.

Now clicking on the link will open your default browser and load the link for
you which is good enough for now.

Notice that the URL is displayed using a monospace font.


\subsection{Links with description}
\href{http://en.wikibooks.org/wiki/LaTeX/Hyperlinks}{\LaTeX{} Hyperlinks}

Being able to link to other documents online is nice, but sometimes it would
be more desirable to have a piece of text describing the link as the
interactive link.

This is accomplished using the \verb`\href{url}{description}` command, which
takes your link's address as first argument, and the description as second.

In this example, \textbf{\LaTeX{} Hyperlinks} is the description,
and is shown instead of the address in the standard document font, but still
clicking on the text, would have the previous behavior: opening the link in the default browser.

\section{Advanced}

\subsection{email addresses}
\href{mailto:my_address@example.com}{my\_address@example.com}

You can use the mailto URI scheme with your email address in order to create
a link. Clicking the link will open the default email client, with a new
empty message and the address in the \textit{To:} field.

If you want people to be able to click on your email address to send you a
message, all you need to do is prefix your email address with \texttt{mailto:}
like in the example, and rewrite your email address in the description,
escaping special characters with \verb`\` like the underscore in the example.

For more technical information concerning the mailto URI scheme, take a look at
the \href{http://en.wikipedia.org/wiki/Mailto}{mailto page} on wikipedia.


\subsection{Future}
You might have noticed that your table of contents is now interactive, but also
that links have a red boarder around them, rendering the table of content
ugly for on-screen displaying.

We'll see next time how that can be customized using the \verb`\hypersetup`
command in the preamble. You can read more about this in the
\href{http://en.wikibooks.org/wiki/LaTeX/Hyperlinks#Customization}
{customization} section of the Hyperlinks chapter in the \LaTeX wikibook.

We'll also cover next time intra-document references and labels.


\end{document}
