\documentclass{article}

\title{First example}
\author{Anon Ymous}

\begin{document}

\maketitle

\section{Structure}

\subsection{Document structure}

Looking now at the source document, you'll notice it consists of 2 parts:

\begin{itemize}
    \item The preamble contains the document class to use (article in this
    example) and metadata, like the document title and author we defined.
    \item The document content, which is everything between
    \verb`\begin{document}` and \verb`end{document}`.
\end{itemize}

The document content itself is made up of several sections, and subsections,
themselves made up of several paragraphs and structures.


\subsection{Whitespace}

In latex, your focus is your documents structure, and the presentational aspect
is taken care of for you. This brings us to discuss what \textit{space} is in
latex.

\textit{Whitespace} characters, such as blank or tab, are treated as
\textit{space} by latex.
Several consecutive whitespace characters are treated as one \textit{space}.
Whitespace at the start of a line is generally ignored, and a single line
break is treated as \textit{whitespace}.

You'll notice that for latex, a paragraph is a sequence of non-blank lines, and
a new paragraph begins right after a blank line: An empty line between two
lines of text defines the end of a paragraph.

Paragraph formatting is taken care of by latex, and that include the
hyphenation of text. If you look at the source, you'll notice that latex
doesn't necessarily break lines where there is a line break in the source
document. For example take a look at the last paragraph: It was written on one
line in the source document, and yet latex formats it beautifully in the pdf.

Several empty lines are treated the same as one empty line.


\section{second section}

\subsection{Verbatim environment}

If you want to introduce text that won't be interpreted by the compiler, you
can use the verbatim environment. Everything written in that environment is
processed as if by a typewriter:

\begin{itemize}
    \item All spaces and line breaks are reproduced verbatim, and the text is
    displayed in a monospace font.
    \item Any latex command in the verbatim environment will be ignored and
    handled as plain text. The reason is that once in the verbatim environment,
    the only command that will be recognized is \verb`\end{verbatim}`.
\end{itemize}

If you want to introduce a short inline verbatim text, you can use the 
\verb`\verb` command:

\verb+raw   text+

Following the \verb`\verb` command, the text must be enclosed in delimiters,
like the ``+" we used here. The delimiter can be any character you like
except *. This will print verbatim all the text within the delimiters.

For example, the code:

\verb|\textbf{This is  printed   verbatim    !}|

And for the verbatim environment:

\begin{verbatim}
\subsection{Second subsection of second section}
Vestibulum   aliquam            lectus       hendrerit urna mattis pellentesque. Cras dignissim
egestas sem, ac pulvinar neque
venenatis ac.
\end{verbatim}

\subsection{Third subsection of second section}

Lorem ipsum dolor sit amet, consectetur adipiscing elit. Suspendisse at dolor nibh, eget tempus neque. Vestibulum ante ipsum primis in faucibus orci luctus et ultrices posuere cubilia Curae; Cras sapien arcu, aliquam ac auctor ac, venenatis vel lectus. Donec posuere mauris eget dui malesuada eu tristique neque ultricies.

\end{document}
