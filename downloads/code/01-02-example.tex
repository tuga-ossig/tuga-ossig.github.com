\documentclass[a4paper]{book}

\usepackage[utf8]{inputenc}
\usepackage[T1]{fontenc}
\usepackage[ngerman]{babel}

\author{Mickey Mouse}
\date{\today}

\title{second example}

\begin{document}

\maketitle
\tableofcontents


\chapter{Not so Simple}

As you can see, this example is a bit more elaborate, and will allow us to
introduce \textbf{latex commands}, \textbf{latex packages}, and the
\textbf{latex table of content}.


\section{Commands and Parameters}

Latex commands are introduced by a backslash, and possibly have parameters,
which might be enclosed in curly braces.

Some commands come in pairs, with a beginning command, the corresponding ending
command, and the content encapsulated in between.

Following \verb`\commandname`, there may be one or more parameters. Optional
parameters are enclosed in square brackets. Required parameters are enclosed
in curly braces.

So the structure of a command in latex looks like this:\\
\verb`\commandname[option1, option2]{parameter1, parameter2}`

\verb`\documentclass` is a command too, and as you can see, here we define the
optional paper size to be A4.


\section{Packages}

If latex's default behavior, and processing isn't suitable for you, it's
possible for you to customize it through some changes. When those changes are
frequently useful, you can move those changes into what latex calls packages.
That helps keeping your documents uncluttered with changes not related to the
content.

This simplifies the process of distributing those changes so that others having
the same need won't have to go through the trouble you already went through.
All they have to do is download your package, and then use the package.

This works in the other direction too: if you have a need, and somebody already
published a package for it, you can grab it and use it.
This is what we do in this document's preamble with the three \verb`\usepackage`
latex commands.

The first 2 setup the font and text encoding. You'll want to have them in every
document you write.

The third command asks latex to use the package \textit{babel} with the option
\textit{ngerman}.
It is used to tell latex in which language the document will be, here
\textit{ngerman} means new German. The language setting defines certain things
for you.
The \verb`\chapter` command for instance, renders as \textit{Kapitel}, and the
\verb`\tableofcontents` as \textit{Inhaltsverzeichnis}.


\chapter{second chapter}


\section{Document title and Table of contents}

The command \verb`\maketitle` generates a title page for you from the document
metadata, also called \textbf{top matter} in \LaTeX.

It's inserted for you in the resulting document right where the command is
called in the source document.

\verb`\tableofcontents` generates a table of contents for you from all the
headers in the document, and inserts it for you in the resulting document right
where the command is called in the source document.


\end{document}
