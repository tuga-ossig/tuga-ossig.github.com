\documentclass[handout]{beamer}

\usepackage[utf8]{inputenc}
\usepackage[T1]{fontenc}
\usepackage[ngerman]{babel}

\usepackage{amsmath}
\usepackage{amssymb}

\usepackage{hyperref}

\author{Anon Ymous}
\date{\today}

\title{Beamer slides}

\begin{document}
\begin{frame}
    \frametitle{There Is No Largest Prime Number}
    \framesubtitle{The proof uses \textit{reductio ad absurdum}.}
    \begin{theorem}
        There is no largest prime number.
    \end{theorem}
    \begin{proof}
        \begin{enumerate}
            \item<1-| alert@1> Suppose $p$ were the largest prime number.
            \item<2-> Let $q$ be the product of the first $p$ numbers.
            \item<3-> Then $q+1$ is not divisible by any of them.
            \item<1-> But $q + 1$ is greater than $1$, thus divisible by some prime
            number not in the first $p$ numbers.\qedhere
        \end{enumerate}
    \end{proof}
\end{frame}
\begin{frame}
    \frametitle{Beamer terminology}
    \framesubtitle{frame}
    \begin{description}
        \item<1-| alert@1>[frame] is basic building blocks of presentations.
        \item<2-| alert@2>[frame] consists of a series of slides.
        \item<3-| alert@3>[frame] is a beamer environment.
    \end{description}
\end{frame}
\begin{frame}
    \frametitle{More terminology}
    \framesubtitle{frametitle, framesubtitle}
    \begin{description}
        \item<1-| alert@1>[frametitle] Title displayed on the frame.
        \item<2-| alert@2>[framesubtitle] Subtitle displayed on the frame.
    \end{description}
\end{frame}
\begin{frame}
    \frametitle{overlays}
    \begin{block}{Definition}
        \alert{Overlays} are the equivalent of PowerPoint transitions in beamer.
        Allows elements to be shown on different slides in the same frame.
    \end{block}
\end{frame}


\end{document}
